\chapter{Implementierung}

Im Folgenden wird vorgestellt, wie das beschriebene Verfahren, welches wir auf Basis der vorigen
\"Uberlegungen erhalten haben, konkret in Programmcode umgesetzt wird. Aufgrund des hohen Ma{\ss}es an Kontrolle \"uber den Einsatz der
Hardware, des damit einhergehenden Geschwindigkeitsvorteils, sowie der erleichterten Anbindung an bestehende Frameworks, wurde zu
diesem Zweck die Programmiersprache C++ gew\"ahlt. Der funktionale Kern des Programms wird hierbei als \textit{statische Bibliothek}
realisiert, diese kann dann zum konkreten Einsatz bequem eingebunden werden.

\section{Aufbau der Software}



\subsection{Klassendesign}

\subsection{Benutzungsschnittstelle}

\section{Verwendete Entwicklerwerkzeuge}

\subsection{CMake}

\subsection{Google Test}

\section{Visualisierung und Testl\"aufe}
