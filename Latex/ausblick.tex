\chapter{Ausblick}

Selbstverst\"andlich gilt es noch einige Elemente zu ber\"ucksichtigen, bevor das Verfahren produktiv eingesetzt werden kann.
In einem echten Versuchsaufbau herrscht beispielsweise kein echtes Vakuum vor. Es kann also passieren, dass das Teilchen auf seinem
Weg mit anderen Teilchen zusammentrifft, wodurch Wechselwirkungen entstehen. Diese Ereignisse, welche
sich nicht vorhersehen lassen, k\"onnen mit den gegebenen Auftrittswahrscheinlichkeiten in das Verfahren integriert werden. So kann beispielsweise festgestellt werden, wie sich die Teilchenflugbahn im Mittel ver\"andert,
wenn der Ladungstr\"ager ein anderes Material durchquert oder beispielsweise mit anderen Teilchen wechselwirkt.

Ein weiterer Aspekt, der noch besondere Betrachtung verdient, ist die Behandlung des Approximationsfehlers. Bisher gilt die Faustregel, dass
der Fehler abnimmt, je feiner das Magnetfeld unterteilt wird. Unter Umst\"anden sind jedoch konkrete Fehlerschranken von Interesse, die bei der
Simulation eingehalten werden sollen. Ein weiterer Schritt in diese Richtung ist die Konstruktion eines adaptiven Verfahrens auf Basis dieses Algorithmus.
Hier setzt man sich zum Ziel, die Unterteilung an die \"Anderung des magnetischen Feldes anzupassen. Besteht beispielsweise eine
gro{\ss}e \"Anderung des Feldes, so wird die Diskretisierung sehr feingranular vorgenommen, bei ann\"ahernd homogenen Feldern kann
eine gr\"obere Unterteilung verwendet werden. Dies lie{\ss}e sich beispielsweise realisieren, indem die Simulation mit
verschiedenen Zeitschritten durchgef\"uhrt wird. Durch einen Vergleich der Abweichungen mit einem zuvor spezifizierten
Toleranzkriterium l\"asst sich erkennen, wo eine Verfeinerung erforderlich ist, und wo der gr\"obere Zeitschritt ausreicht.
So kann an den betreffenden Stellen immer weiter verfeinert werden, bis die Genauigkeit den Anforderungen in jedem Punkt gen\"ugt.

Gleichwohl liegt mit dem beschriebenen Verfahren eine solide Basis vor, auf der zahlreiche weitere Schritte aufbauen k\"onnen. Interessant
wird vor allem die Einbindung in bestehende Frameworks sein, die mithilfe dieser Simulationen Experimente der Praxis evaluieren und bewerten.
Vor allem die Performance spielt hier eine herausragende Rolle, da mit sehr gro{\ss}en Datenmengen gearbeitet wird. In dieser Hinsicht ist der
Algorithmus auf der Grundlage von C++ jedoch bestens gewappnet, sich auch  mit anderen konkurrierenden Verfahren zu messen.
