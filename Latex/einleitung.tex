\chapter{Einleitung}

Bevor ein physikalisches Experiment in der Praxis durchgef\"uhrt werden kann, steht zu Beginn stets eine
pr\"azise Simulation der m\"oglichen Ereignisse. Egal ob am Teilchenbeschleuniger des CERN oder am
Institut f\"ur Kernphysik des Forschungszentrums in J\"ulich, f\"ur einen erfolgreichen Versuch
m\"ussen schon im Voraus alle Eventualit\"aten ber\"ucksichtigt werden. Zu diesem Zweck ist die
computergest\"utzte Simulation ein m\"achtiges Werkzeug in den H\"anden der Physiker. Durch das
wiederholte Durchspielen des Versuchs kann \"uber jedes Detail entschieden werden, noch bevor der
Versuchsaufbau in der Praxis umgesetzt wird.

Das Ziel dieser Arbeit ist es nun, die Entstehung eines
solchen Verfahrens zu beschreiben, welches ein bestimmtes Element dieses Simulationsprozesses
herausgreift: Die Simulation der Flugbahnen bewegter Ladungstr\"ager in magnetischen Feldern.
In vielen gro{\ss}en Versuchsaufbauten werden magnetische
Felder eingesetzt, um die freien Ladungstr\"ager auf bestimmte Bahnen zu lenken, oder um
beispielsweise durch das Flugverhalten der Teilchen R\"uckschl\"usse auf deren Beschaffenheit zu
ziehen. Diese Arbeit legt die Basis f\"ur ein Verfahren, welches die Simulation dieser Vorg\"ange
erm\"oglicht.

Zu diesem Zweck werden zun\"achst die physikalischen Grundlagen beschrieben, die f\"ur das
Verst\"andnis dieser Arbeit unerl\"asslich sind. Aufbauend auf diesem Fundament wird im weiteren
Verlauf die Berechnung der Teilchenflugbahn erl\"autert. Der Weg f\"uhrt hierbei zun\"achst \"uber
den leicht verst\"andlichen Einzelfall, um dann durch wirkungsvoll gew\"ahlte Basistransformationen immer weiter
hin zur Verallgemeinerung auf beliebige Konstellationen von Teilchenzust\"anden und Magnetfeldern zu
gelangen. Abschlie{\ss}end wird nach einem \"Uberblick \"uber das resultierende Verfahren die
konkrete Implementierung in der Programmiersprache C++ vorgestellt. Diese Gelegenheit wird
au{\ss}erdem dazu genutzt, um einige Werkzeuge zu demonstrieren, die sich bei der Entwicklung als
\"au{\ss}erst wertvoll erwiesen haben.

Die Funktionsf\"ahigkeit des Verfahrens wird schlussendlich
durch die Pr\"asentation einiger Fallbeispiele gezeigt, welche die Ergebnisse der Simulationen in
konkreten Situationen visuell untermauern.
