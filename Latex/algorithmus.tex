\chapter{Aufbau des Verfahrens}

Um nun aufbauend auf diesen Erkenntnissen die allgemeinen Flugbahnen beliebiger Ladungstr\"ager durch beliebige, unter Umst\"anden
auch \textit{inhomogene}, Magnetfelder simulieren zu k\"onnen, sind mehrere Zwischenschritte notwendig.
Unter Verwendung der Ergebnisse aus \fref{sec:flugbahnbeschreibung} wird zun\"achst ein Verfahren hergeleitet, welches es erm\"oglicht,
die Teilchenflugbahnen in \textit{homogenen} Magnetfeldern zu simulieren. Dieses Verfahren kann anschlie{\ss}end auf beliebige
Magnetfelder ausgeweitet werden, indem man diese abschnittsweise durch homogene Teilfelder approximiert, auf welchen dann das
hergeleitete Verfahren zur Anwendung kommt.

\section{Simulation der Flugbahn im homogenen Magnetfeld}

Bewegen sich Teilchen im Raum, so kann dies in allen m\"oglichen Orientierungen erfolgen. Um diesem Sachverhalt Rechnung zu tragen,
werden zun\"achst s\"amtliche Schritte im lokalen Bezugssystem des Ladungstr\"agers durchgef\"uhrt. Unabh\"angig von der speziellen
Orientierung des Teilchens gew\"ahrleistet dies einheitliche Rechenvorschriften, welche im Anschluss durch einfache
Basistransformationen in das globale Koordinatensystem \"uberf\"uhrt werden k\"onnen.

\subsection{Die Flugbahn im lokalen Bezugssystem des Ladungstr\"agers}

Das lokale Koordinatensystem wird nun so gew\"ahlt, dass sich die Position des Teilchens genau im Ursprung befindet.
Weiterhin zeigen die magnetischen
Feldlinien genau in Richtung der \textit{z-Achse}, die orthogonale Projektion der Teilchenbewegung auf die xy-Ebene zeigt genau
entlang der \textit{y-Achse}. Auf diese Weise lassen sich die beiden Anteile der Bewegung, welche in \fref{sec:flugbahnbeschreibung}
beschrieben wurden, optimal trennen: Der gleichf\"ormige Anteil der Bewegung, welcher auf \(v_\parallel\) zur\"uckzuf\"uhren ist,
verl\"auft genau entlang der \textit{z-Achse}, w\"ahrend \(v_\perp\) eine Kreisbewegung in der \textit{xy-Ebene} verursacht.
\(v_\parallel\) beeinflusst also nur die \textit{z-Koordinate} des Ladungstr\"agers, w\"ahrend \(v_\perp\) ausschlie{\ss}lich die
\textit{x}-, sowie die \textit{y-Koordinate} beeinflusst.

Um nun den Mittelpunkt \(M\) des Kreises zu ermitteln, welchem die Bewegung in der \textit{xy-Ebene} folgt, kann zun\"achst der
Radius \(r\) des Kreises bestimmt werden. Gem\"a{\ss} der Konstruktion und unter Ber\"ucksichtigung der \textit{Linke-Hand-Regel}
liegt \(M\) dann genau auf der \textit{x-Achse} und hat die xy-Koordinaten \(\left(-r| 0 \right)\). Der Radius l\"asst sich nun,
wie in \fref{sec:flugbahnbeschreibung} mit \fref{eq:radius} beschrieben, ohne weiteres berechnen. Man beachte, dass aufgrund der
Konstruktion nur die \textit{z-Koordinate} des Magnetfeldes \(\vec{B}\), sowie die \textit{y-Koordinate} der Teilchengeschwindigkeit
\(\vec{v}\) relevant ist. Weiterhin ist ebenfalls konstruktionsbedingt \(\alpha = \frac{\pi}{2}\), weshalb der Term \(\sin{\alpha}\)
entf\"allt. Benutzen wir also \(v_\perp = \vec{v}.y\) und \(B = \vec{B}.z\), so erhalten wir f\"ur den Radius \(r\) im lokalen
Bezugssystem:
\begin{equation*}
  r = \frac{m \cdot v_\perp}{Q \cdot B}
\end{equation*}
Dies entspricht exakt \fref{eq:radius} und wird hier nur aus Gr\"unden der Lesbarkeit erneut aufgef\"uhrt.

\subsection{\"Ubertragung auf das globale Koordinatensystem}

\section{Ausweitung auf inhomogene Magnetfelder}

\subsection{Diskretisierung inhomogener Magnetfelder}

\subsubsection{TODO Fehler}

\section{Abbruchkriterien}

\section{Der resultierende Algorithmus}
