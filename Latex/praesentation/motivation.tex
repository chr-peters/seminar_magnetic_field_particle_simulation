%\chapter{Einleitung}
\mode*
\begin{frame}
  \frametitle{Motivation}
  \begin{itemize}
    \item Planung physikalischer Experimente
    \begin{itemize}
      \item Welche Ereignisse k\"onnen auftreten?
      \item Kann der Versuch in der Praxis durchgef\"uhrt werden?
    \end{itemize}
    \item Computergest\"utzte Simulation liefert Antworten
      \begin{itemize}
        \item Erm\"oglicht das wiederholte Durchspielen eines Versuchs
      \end{itemize}
    \item Enormes finanzielles Einsparpotential
      \begin{itemize}
        \item Die Zeiten von "`trial and error"' sind vorbei
        \item Ein Versuch wird nur dann in der Praxis durchgef\"uhrt,
          wenn er vorher alle Simulationstests bestanden hat
      \end{itemize}
    \item Wie entsteht ein solches Verfahren?
  \end{itemize}
\end{frame}

\begin{frame}
  \frametitle{Motivation}
  \begin{itemize}
    \item Magnetische Felder werden in vielen gro{\ss}en
      Versuchsaufbauten eingesetzt (CERN, IKP am FZJ, \ldots)
    \begin{itemize}
      \item lenken freie Ladungstr\"ager auf bestimmte Bahnen
      \item erm\"oglichen R\"uckschl\"usse auf Teilchenbeschaffenheit
        durch Analyse des Flugverhaltens
    \end{itemize}
    \item Wie k\"onnen diese Vorg\"ange simuliert werden?
  \end{itemize}
\end{frame}
\mode<all>
