\mode*
\begin{frame}
  \frametitle{Physikalische Grundlagen}
  \textbf{Die Lorentz-Kraft}
  \begin{itemize}
    \item proportional zur Ladung \(Q\) des Teilchens
    \item proportional zur Geschwindigkeit \(v\)
    \item senkrecht zur Bewegungsrichtung
    \item senkrecht zur Magnetfeldrichtung
    \item \"andert sich abh\"angig vom Winkel \(\alpha\) zwischen Teilchenbewegung und Magnetfeld
    \item Wirkungsrichtung? \(\rightarrow\)Linke-Hand-Regel
  \end{itemize}
  \vspace*{\fill}
  \begin{equation*}
    \label{eq:lorentz_abs}
    F_L = Q \cdot v \cdot B \cdot \sin{\alpha}
  \end{equation*}
  \vspace*{\fill}
\end{frame}

\begin{frame}
  \frametitle{Physikalische Grundlagen}
  \textbf{Die gleichf\"ormige Kreisbewegung}
  \begin{itemize}
    \item K\"orper muss durch Zentripetalkraft \(F\) auf Bahn gehalten werden
    \item \(F\) ist radial zum Mittelpunkt des Kreises hin gerichtet
  \end{itemize}
  \vspace*{\fill}
  \begin{equation*}
    \label{eq:zentripetalkraft}
    F = \frac{mv^2}{r} = m\omega^2r
  \end{equation*}
  \vspace*{\fill}
  \begin{itemize}
    \item \(m\) ist die Masse des K\"orpers
    \item \(v\) ist die Bahngeschwindigkeit
    \item \(\omega\) ist die Winkelgeschwindigkeit
    \item \(r\) ist der Radius der Kreisbahn
  \end{itemize}
\end{frame}

\begin{frame}
  \frametitle{Physikalische Grundlagen}
  \textbf{Beschreibung der Flugbahn}
  \begin{itemize}
    \item Unterteilung von \(v\) in \(v_\perp\) und \(v_\parallel\)
    \item \(v_\parallel\) von Lorentz-Kraft unbeeinflusst, daher gleichf\"ormige Bewegung
      \begin{itemize}
        \item \(S = v_\parallel \cdot t\)
      \end{itemize}
    \item auf \(v_\perp\) wirkt Lorentz-Kraft wie Zentripetalkraft \(\rightarrow\) Kreisbewegung
    \item durch Gleichsetzen und Umformen ergeben sich folgende Zusammenh\"ange:
  \end{itemize}
  \begin{equation*}
    \label{eq:radius}
    r = \frac{m \cdot v_\perp}{Q \cdot B}
  \end{equation*}
  \begin{equation*}
    \label{eq:omega}
    \omega = \frac{Q \cdot B}{m}
  \end{equation*}
\end{frame}
\mode<all>
