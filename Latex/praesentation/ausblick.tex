\mode*

\begin{frame}
  \frametitle{Ausblick}
  \onslide<+->
  \begin{itemize}
    \item<+-> Das Verfahren ist noch nicht perfekt
      \begin{itemize}
        \item<+-> echter Versuchsaufbau: kein perfektes Vakuum
        \item<+-> Wechselwirkungen mit anderen Teilchen sind m\"oglich
        \item<+-> Wie verh\"alt sich die Flugbahn bei anderen Materialien?
          \begin{itemize}
            \item<+-> Ansatz: Berechnung einer "`mittleren Abweichung"' basierend auf stochastischen Zusammenh\"angen
          \end{itemize}
      \end{itemize}
    \item<+-> Behandlung des Approximationsfehlers
      \begin{itemize}
        \item<+-> konkrete Fehlerschranken sind von Interesse
        \item<+-> Ansatz: Konstruktion eines adaptiven Verfahrens
          \begin{itemize}
            \item<+-> Verfeinerung der Schrittweite bis ein Toleranzkriterium erf\"ullt ist
            \item<+-> starke Feld\"anderung \(\rightarrow\) feinere Unterteilung als bei geringerer \"Anderung
          \end{itemize}
      \end{itemize}
    \item<+-> Gleichwohl: Solide Basis
      \begin{itemize}
        \item<+-> weitere Schritte k\"onnen folgen
        \item<+-> mit C++ gut gewappnet f\"ur die Einbindung in bestehende Simulationsframeworks
      \end{itemize}
  \end{itemize}
\end{frame}

\begin{frame}
  \frametitle{Vielen Dank!}
  \centering
  \textbf{Vielen Dank f\"ur eure Zeit!}
\end{frame}

\mode<all>
