\chapter{Physikalische Grundlagen}

Um ein Fundament zu schaffen, auf welchem diese Arbeit im weiteren Velauf aufbauen kann, werden
an dieser Stelle zun\"achst die physikalischen Grundlagen dargelegt, auf denen das sp\"ater beschriebene
Verfahren basiert. Es werden vornehmlich die Wechselwirkungen von bewegten Teilchen in homogenen
Magnetfeldern beschrieben, ein \"Uberblick dieser Zusammenh\"ange ist f\"ur das weitere Verst\"andnis
dieser Arbeit unbedingt notwendig. Die hier dargelegten Ausf\"uhrungen berufen sich im Wesentlichen auf \cite{Vog99}.

\section{Die Lorentz-Kraft und ihre Eigenschaften}

Bewegt sich ein geladenes Teilchen mit der Geschwindigkeit \(\vec{v}\) durch ein magnetisches Feld \(\vec{B}\),
so erf\"ahrt es die Lorentz-Kraft \(\vec{F_L}\). Diese Kraft wirkt nur, wenn sich das geladene Teilchen \textit{bewegt}, weiterhin
ist der Betrag dieser Kraft sowohl proportional zur Ladung \(Q\) des Teilchens, als auch zu seiner Geschwindigkeit.
Die Lorentz-Kraft wirkt stets sowohl \textit{senkrecht} zur Bewegungsrichtung des Teilchens, als auch \textit{senkrecht} zur Richtung
des Magnetfeldes. Das Teilchen wird wird in seiner Flugbahn also
\textit{seitlich} abgelenkt. Der Betrag der Kraft h\"angt au{\ss}erdem von der Richtung der Teilchenbewegung ab, verl\"auft diese
parallel zur Richtung des magnetischen Feldes, so gilt \(\abs{\vec{F_L}} = 0\). F\"ur beliebige Winkel \(\alpha\) ver\"andert
sich der Betrag der Lorentz-Kraft wie \(\abs{\vec{v}}\sin{\alpha}\).
Fasst man alle diese Eigenschaften zusammen, so erh\"alt man den folgenden Ausdruck:
\begin{equation}
  \vec{F_L} = Q \cdot \vec{v} \times \vec{B}
\end{equation}
Ist man nur am Betrag der Lorentz-Kraft interessiert, so l\"asst sich dessen Berechnung durch den folgenden Ausdruck vereinfachen:
\begin{equation}
  \abs{\vec{F_L}} = \abs{Q} \cdot \abs{\vec{v}} \cdot \abs{\vec{B}} \cdot \sin{\alpha}
\end{equation}

\section{Bewegte Ladungstr\"ager im homogenen Magnetfeld}

Jetzt wird es schon etwas konkreter
