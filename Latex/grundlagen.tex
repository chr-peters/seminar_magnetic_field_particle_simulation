\chapter{Physikalische Grundlagen}

Um ein Fundament zu schaffen, auf welchem diese Arbeit im weiteren Velauf aufbauen kann, werden
an dieser Stelle zun\"achst die physikalischen Grundlagen dargelegt, auf denen das sp\"ater beschriebene
Verfahren basiert. Es werden vornehmlich die Wechselwirkungen von bewegten Teilchen in homogenen
Magnetfeldern beschrieben, ein \"Uberblick dieser Zusammenh\"ange ist f\"ur das weitere Verst\"andnis
dieser Arbeit unbedingt notwendig. Die hier dargelegten Ausf\"uhrungen berufen sich im Wesentlichen auf \cite{Vog99}
\footnote{Vgl. S.354-355 zur Lorentz-Kraft und S.451-453 zu Elektronen in homogenen Magnetfeldern.}.

\section{Die Lorentz-Kraft und ihre Eigenschaften}

Bewegt sich ein geladenes Teilchen mit der Geschwindigkeit \(v\) durch ein magnetisches Feld \(B\),
so erf\"ahrt es die Lorentz-Kraft \(F_L\). Diese Kraft wirkt nur, wenn sich das geladene Teilchen \textit{bewegt}, weiterhin
ist der Betrag dieser Kraft sowohl proportional zur Ladung \(Q\) des Teilchens, als auch zu seiner Geschwindigkeit.
Die Lorentz-Kraft wirkt stets sowohl \textit{senkrecht} zur Bewegungsrichtung des Teilchens, als auch \textit{senkrecht} zur Richtung
des Magnetfeldes. Das Teilchen wird in seiner Flugbahn also
\textit{seitlich} abgelenkt. Der Betrag der Kraft h\"angt au{\ss}erdem von der Richtung der Teilchenbewegung ab: Verl\"auft diese
parallel zur Richtung des magnetischen Feldes, so gilt \(F_L = 0\). F\"ur beliebige Winkel \(\alpha\) ver\"andert
sich der Betrag der Lorentz-Kraft wie \(v \cdot \sin{\alpha}\).
Fasst man alle diese Eigenschaften zusammen, so erh\"alt man den folgenden Ausdruck:
\begin{equation}
  \label{eq:lorentz_abs}
  F_L = Q \cdot v \cdot B \cdot \sin{\alpha}
\end{equation}
Die konkrete Wirkungsrichtung der Lorentz-Kraft l\"asst sich leicht anhand der \textit{Linke-Hand-Regel} ermitteln:
Zeigt der Daumen der linken Hand in Richtung der Ursache f\"ur die Lorentz-Kraft, also der Bewegung eines geladenen Teilchens,
und zeigt ferner der Zeigefinger der linken Hand in Richtung des magnetischen Feldes, so ergibt sich die Richtung der
Lorentz-Kraft aus der Richtung des Mittelfingers der linken Hand. Fasst man die bisherigen Gr\"o{\ss}en als die Vektoren
\(\vec{v}\), \(\vec{B}\) und \(\vec{F_L}\) auf, so bilden diese ein \textit{Linkssystem}.
Hierbei einigt man sich auf die sogenannte \textit{physikalische Stromrichtung}: Negative Ladungen (beispielsweise
freie Elektronen) bewegen sich per Definition vom negativen hin zum positiven Pol einer Spannungsquelle. Im Falle der
\textit{Linke-Hand-Regel} zeigt der Daumen also immer hin zur \textit{physikalischen Stromrichtung}.
W\"urde man unter sonst gleichen Bedingungen die Richtung des Stromes andersherum definieren (\textit{technische Stromrichtung}),
so k\"ame anstelle der linken Hand nun die rechte Hand zur Anwendung. Alle Gr\"o{\ss}en blieben aber unver\"andert, somit bleibt es
jedem selbst \"uberlassen, welche der Definitionen er bevorzugt.

\section{Bewegte Ladungstr\"ager im homogenen Magnetfeld}

Bewegt sich nun ein Ladungstr\"ager durch ein homogenes Magnetfeld, so stellt sich die Frage, durch welche Eigenschaften sich
die Flugbahn beschreiben l\"asst, auf welcher sich dieser aufgrund der Lorentz-Kraft bewegt. Beachtet man die Tatsache, dass die
Lorentz-Kraft stets senkrecht zur Bewegungsrichtung wirkt, so wird man unweigerlich zu der Schlussfolgerung gelangen, dass es sich
hierbei um eine gleichf\"ormige Kreisbewegung handeln muss. Um dies nachvollziehen zu k\"onnen, werden die allgemeinen Eigenschaften
einer gleichf\"ormigen Kreisbewegung und die Eigenschaften der Flugbahn des Ladungstr\"agers im Speziellen im Folgenden n\"aher
beschrieben.

\subsection{Eigenschaften der gleichf\"ormigen Kreisbewegung}
\label{sec:kreisbewegung}

Bewegt sich ein K\"orper der Masse \(m\) mit konstanter Geschwindigkeit \(v\) auf einer kreisf\"ormigen Bahn mit dem Radius \(r\) um den
Kreismittelpunkt \(M\), so spricht man von einer gleichf\"ormigen Kreisbewegung. Der Begriff "`gleichf\"ormig"' r\"uhrt daher,
dass sich der Betrag der \textit{Bahngeschwindigkeit} \(v\) des K\"orpers nicht \"andert, wohl aber die Richtung, welche stets
tangential zur Kreisbahn verl\"auft. W\"ahrend die \textit{Bahngeschwindigkeit} \(v\) angibt, welche Bogenl\"ange des Kreises in einer
bestimmten Zeit \(t\) durchlaufen wird, so beschreibt die \textit{Winkelgeschwindigkeit} \(\omega\), welcher Winkel (man verwendet hier
das \textit{Bogenma{\ss}}) in \(t\) zur\"uckgelegt wird.

Damit sich ein K\"orper \"uberhaupt auf einer Kreisbahn bewegen kann, muss auf ihn eine Kraft \(F\) wirken, die ihn auf dieser Bahn
h\"alt. Diese Kraft, die stets radial zum Mittelpunkte des Kreises hin gerichtet ist, nennt man \textit{Zentripetalkraft}.
Es gilt der folgende Zusammenhang f\"ur den Betrag der Zentripetalkraft:
\begin{equation}
\label{eq:zentripetalkraft}
F = \frac{mv^2}{r} = m\omega^2r
\end{equation}

\subsection{Beschreibung der Flugbahn}
\label{sec:flugbahnbeschreibung}

Betrachtet man vor diesem Hintergrund erneut die Lorentz-Kraft \(F_L\), so wird deutlich, dass diese alle Eigenschaften der in
\fref{sec:kreisbewegung} beschriebenen Zentripetalkraft erf\"ullt. Dies liegt daran, dass die Lorentz-Kraft zum einen den Betrag
der Teilchengeschwindigkeit nicht \"andert und zum anderen immer senkrecht auf der Bewegungsrichtung des Teilchens steht.

Da sich der Ladungstr\"ager jedoch im Allgemeinen nicht vollst\"andig senkrecht zum Magnetfeld bewegt, gibt es auch eine Komponente
der Geschwindigkeit des Teilchens (im Folgenden \(v_\parallel\)), welche von der Lorentz-Kraft unbeeinflusst bleibt, da sie parallel zum
Magnetfeld verl\"auft.
Um diesem Sachverhalt gerecht zu werden, werden diese unterschiedlichen Komponenten getrennt voneinander betrachtet. Dies hat zur
Folge, dass der Anteil der Bewegungsrichtung, welcher senkrecht zum magnetischen Feld verl\"auft (\(v_\perp\)), eine Kreisbewegung
beschreibt,
der Anteil \(v_\parallel\) wird jedoch nicht von der Lorentz-Kraft beeinflusst. 
Die Summe dieser beiden Bewegungen ergibt
eine Schraubenlinie, welche durch einen Zylinder mit dem Radius \(r\) der Kreisbahn begrenzt ist. Die orthogonale Projektion dieser
Schraubenbewegung vollzieht genau die Kreisbewegung, welche von der Lorentz-Kraft verursacht wird.
Um die konkreten Eigenschaften der Flugbahn zu quantifizieren, reicht es also aus, beide Teile der Bewegung f\"ur sich genommen
zu beschreiben. 

Betrachtet man die zum Magnetfeld senkrechte Komponente \(v_\perp\) der Teilchenbewegung, so l\"asst sich die resultierende Kreisbewegung
eindeutig durch die Angabe des Radius \(r\), sowie durch die Winkelgeschwindigkeit \(\omega\) beschreiben. Den Radius \(r\) erh\"alt
man aus dem Zusammenhang, dass die Lorentz-Kraft die Rolle der Zentripetalkraft einnimmt. Setzt man \fref{eq:lorentz_abs} unter der
Bedingung \(\alpha = \frac{\pi}{2}\), die sich aus der Konstruktion ergibt (\(v_\perp\) steht \textit{senkrecht} zum Magnetfeld),
mit \fref{eq:zentripetalkraft}, der Gleichung der Zentripetalkraft, gleich, so erh\"alt man:
\begin{equation*}
  \frac{m \cdot v_\perp^2}{r} = Q \cdot v_\perp \cdot B
\end{equation*}
\begin{equation}
  \label{eq:radius}
  r = \frac{m \cdot v_\perp}{Q \cdot B}
\end{equation}
Ein \"ahnlicher Ansatz wird verwendet, um die Winkelgeschwindigkeit \(\omega\) der Kreisbewegung zu berechnen. Der einzige Unterschied
ist, dass die Gleichung der Zentripetalkraft verwendet wird, welche die Winkelgeschwindigkeit enth\"alt. Gleichsetzen ergibt den
folgenden Ausdruck:
\begin{equation*}
  m \cdot \omega^2 \cdot r = Q \cdot v_\perp \cdot B
\end{equation*}
\begin{equation*}
  \omega^2 = \frac{Q \cdot v_\perp \cdot B}{m \cdot r}
\end{equation*}
Setzen wir nun f\"ur \(r\) das Ergebnis aus \fref{eq:radius} ein, so ergibt sich folgender Zusammenhang f\"ur die
Winkelgeschwindigkeit der Bewegung:
\begin{equation*}
  \omega^2 = \frac{Q^2 \cdot v_\perp \cdot B^2}{m^2 \cdot v_\perp} = \frac{Q^2 \cdot B^2}{m^2}
\end{equation*}
\begin{equation}
  \omega = \frac{Q \cdot B}{m}
\end{equation}

Um den Bewegungsanteil \(v_\parallel\) zu beschreiben, welcher entlang des Magnetfeldes verl\"auft, gen\"ugen die Gesetze einer
gleichf\"ormigen Bewegung konstanter Geschwindigkeit. Ist man daran interessiert, welche Strecke \(S\) das Teilchen innerhalb einer
Zeit \(t\) zur\"ucklegt, so gilt der folgende allgemein bekannte Zusammenhang:
\begin{equation}
  S = v_\parallel \cdot t
\end{equation}
Addiert man diese beiden unabh\"angigen Teile der Bewegung, so erh\"alt man insgesamt die oben beschriebenen Schraubenlinien, welche
das Teilchen auf seinem Weg durch das homogene Magnetfeld verfolgt.
