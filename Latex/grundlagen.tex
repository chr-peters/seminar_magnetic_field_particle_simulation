\chapter{Physikalische Grundlagen}

Um ein Fundament zu schaffen, auf welchem diese Arbeit im weiteren Velauf aufbauen kann, werden
an dieser Stelle zun\"achst die physikalischen Grundlagen dargelegt, auf denen das sp\"ater beschriebene
Verfahren basiert. Es werden vornehmlich die Wechselwirkungen von bewegten Teilchen in homogenen
Magnetfeldern beschrieben, ein \"Uberblick dieser Zusammenh\"ange ist f\"ur das weitere Verst\"andnis
dieser Arbeit unbedingt notwendig. Die hier dargelegten Ausf\"uhrungen berufen sich im Wesentlichen auf \cite{Vog99}.

\section{Die Lorentz-Kraft und ihre Eigenschaften}

Bewegt sich ein geladenes Teilchen mit der Geschwindigkeit \(\vec{v}\) durch ein magnetisches Feld \(\vec{B}\),
so erf\"ahrt es die Lorentz-Kraft \(\vec{F_L}\). Diese Kraft wirkt nur, wenn sich das geladene Teilchen \textit{bewegt}, weiterhin
ist der Betrag dieser Kraft sowohl proportional zur Ladung \(Q\) des Teilchens, als auch zu seiner Geschwindigkeit.
Die Lorentz-Kraft wirkt stets sowohl \textit{senkrecht} zur Bewegungsrichtung des Teilchens, als auch \textit{senkrecht} zur Richtung
des Magnetfeldes. Das Teilchen wird wird in seiner Flugbahn also
\textit{seitlich} abgelenkt. Der Betrag der Kraft h\"angt au{\ss}erdem von der Richtung der Teilchenbewegung ab, verl\"auft diese
parallel zur Richtung des magnetischen Feldes, so gilt \(\abs{\vec{F_L}} = 0\). F\"ur beliebige Winkel \(\alpha\) ver\"andert
sich der Betrag der Lorentz-Kraft wie \(\abs{\vec{v}}\sin{\alpha}\).
Fasst man alle diese Eigenschaften zusammen, so erh\"alt man den folgenden Ausdruck:
\begin{equation}
  \vec{F_L} = Q \cdot \vec{v} \times \vec{B}
\end{equation}
Ist man nur am Betrag der Lorentz-Kraft interessiert, so l\"asst sich dessen Berechnung unter Verwendung des Zusammenhangs
\(\abs{\vec{v} \times \vec{B}} = \abs{\vec{v}} \cdot \abs{\vec{B}} \cdot \sin{\alpha} \) durch den folgenden Ausdruck vereinfachen:
\begin{equation}
  \abs{\vec{F_L}} = \abs{Q} \cdot \abs{\vec{v}} \cdot \abs{\vec{B}} \cdot \sin{\alpha}
\end{equation}
Die konkrete Wirkungsrichtung der Lorentz-Kraft l\"asst sich leicht anhand der \textit{Linke-Hand-Regel} ermitteln:
Zeigt der Daumen der linken Hand in Richtung der Ursache f\"ur die Lorentz-Kraft, also der Bewegung eines geladenen Teilchens,
und zeigt ferner der Zeigefinger der linken Hand in Richtung des magnetischen Feldes, so ergibt sich die Richtung der
Lorentz-Kraft aus der Richtung des Mittelfingers der linken Hand. Die Vektoren \(\vec{v}\), \(\vec{B}\) und \(\vec{F_L}\)
bilden also ein \textit{Linkssystem}.
Hierbei einigt man sich auf die sogenannte \textit{physikalische Stromrichtung}: Negative Ladungen (beispielsweise
freie Elektronen) bewegen sich per Definition vom negativen hin zum positiven Pol einer Spannungsquelle. Im Falle der
\textit{Linke-Hand-Regel} zeigt der Daumen also immer hin zur \textit{physikalischen Stromrichtung}.
W\"urde man unter sonst gleichen Bedingungen die Richtung des Stromes andersherum definieren (\textit{technische Stromrichtung}),
so k\"ame anstelle der linken Hand nun die rechte Hand zur Anwendung. Alle Gr\"o{\ss}en blieben aber unver\"andert, somit bleibt es
jedem selbst \"uberlassen, welche der Definitionen er bevorzugt.

\section{Bewegte Ladungstr\"ager im homogenen Magnetfeld}

Jetzt wird es schon etwas konkreter
